\documentclass[a4paper,10pt]{article}
\usepackage{graphicx}
\usepackage{url}

% 日本語をサポートするためのパッケージを追加
\usepackage{fontspec}
\usepackage{xeCJK}
\setCJKmainfont{IPAexMincho} % メインフォントを設定(適宜変更可能)

% Page layout
\usepackage[margin=25mm,headheight=5mm,headsep=0mm,footskip=0mm,columnsep=6mm]{geometry}

% No page number
\pagestyle{empty}

% Japanese abstract name
\def\abstractname{あらまし}

% Two column
\twocolumn
\raggedbottom

\begin{document}

\title{予稿タイトル}

% 和文著者名
\author{著者名}

% 和文概要
\begin{abstract}
ここにアブストラクトを書く
\end{abstract}

\maketitle
\thispagestyle{empty}

\section{はじめに}

引用例 \cite{test} を書いてみた.

\section{背景}

\subsection{hoge}
小見出し付きの文章.

\begin{enumerate}
\item 番号付き箇条書き 
\item 番号付き箇条書き
\end{enumerate}

\begin{itemize}
\item 箇条書き
\item 箇条書き
\end{itemize}

%---------------------------------------------

\section{研究目的}

\subsection{hoge}
hogehoge
 
\subsection{fuga}
fugafuga

\section{関連研究}

\section{提案手法}

\section{評価}

\section{考察}

\begin{thebibliography}{9}
\bibitem{test}
test\\
\url{http://www.example.com}
\end{thebibliography}

\end{document}
